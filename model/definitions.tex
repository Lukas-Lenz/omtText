\subsection{Model}

\subsubsection*{Problem}

The problem is assigning to each of $n$ assignment slots one of $m$ components. Components have a type and the component needs to fit that of the assignment slot.

\begin{lstlisting}
Problem

	Tags
	Groups
	ComponentTypes	
	
	Assignments
	  ComponentSlots
	    ComponentType
	    Fixed, Component
	  Optionality
	  Weight
	
	Components
	  ComponentType
	  Groups
	  Tags
	
	Rules
	  AssignmentSet
	  Weight
	  Optionality
	  TopCondition
	    ConditionType
	    Subconditions
	    Attributes
\end{lstlisting}


\subsubsection{Components}

The basic elements of each scheduling problem are the \textbf{components}. In a university timetabling problem these could be events, rooms and teachers.
Each component has a \textit{component type} $\type_C$ and a finite domain $\domain_C$.
Component types can be boolean, discrete and finite numeric types (?) or custom types.
Components are assumed to be unique: $i\neq j \implies c_i\neq c_j$.
The domain of a component is, by default, the set of all values of that type.
A problem specifies a set of groups and a set of tags. Each component has for each tag a corresponding value $tag_{j,i}\in\mathbb{N}_0$. For each group $group_j$ and component $c_i$, it holds that either $group_{j,i}$ or $\neg group_{j,i}$.
As an example, a staff scheduling problem could have a component type $\type_{nurse}$ with the corresponding domain $\domain_{nurse}=\{Anna, Ben, Charlie\}$. 
Component types are called \textit{ordered} if they define a strict weak ordering on their domain.


\subsubsection{Assignments}

\textbf{Assignments} are sub-problems of the scheduling problem, in which a number of components of different types need to be combined: for example, an assignment of a room, a time, a class and students. 

Each assignment $a_i$ defines a sub-problem, in which . It consists of slots $as_{i, 1},...,as_{i,k}$ that each specify a component type $type_{i,j}$, a number of components to fill that slot $\in \{1, k\in\mathbb{N}, n\} \subset \mathbb{N}$. Additionally it specifies whether that slot is optional $optional_{i,j}$ and, if so, a weight $weight_{i,j}$.

An assignment $a_i$ consists of $k\geq 1$ component slots $as_{i,1},...,as_{i,j}$.

Slots can be \textit{fixed}, meaning 

In a nurse rostering problem, an assignment $a_i$ could consist of a 
$type_{i,1} = $
Assignments are called \textit{ordered} if they have at least one component slot of an ordered component type.

In the university course timetabling as presented in ITC19 \cite{muller2018university}, an assignment could consist of a single fixed \textit{course} slot, a single non-optional \textit{roomTime} slot and as many optional \textit{student} slots as the course capacity allows. 

\subsubsection{Condition}
A condition as an object holds information that a .
A set of assignments can be evaluated over a condition. The number of assignments depend on the condition type.

A \textbf{basic condition} is one that does not 

A \textbf{composite condition} is a condition the evaluation of which depends on at least one other condition. Such a chain of conditions will always end with basic 


\subsubsection{Rules}

Additionally to the assignments, a problem has \textbf{rules}, specifying the relationship between individual assignments. 

To apply a condition, they need to be associated with a set of assignments over which they are meant to hold, as well as information necessary for th

A rule connects a chain of conditions to a set of combinations of assignment over which it should be enforced.

When a rule is generated, it must be generated for each viable combination of assignments. 



\subsubsection{Model}

A model is an assignment of components to assignment slots such that no hard rule is violated and the sum of penalties of .

