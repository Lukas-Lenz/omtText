\section{Modelling The Second International Nurse Rostering Competition (2014)}

In short, the scheduling problem presented in
\cite{ceschia2019second} consists of assigning nurses to shifts in multiple consecutive planning periods.
The competition was held in 2015, on the website the benchmarks and a solution validator are still available.

%\subsection{The Second International Nurse Rostering Competition (INRC-II)}

%- stage: one week that will be scheduled at a time
%- scenario: global information that holds across all stages
%- week data: specific data for a single week
%- history: information that must be passed from the last scheduled week to the currently to be scheduled week
%- the solver must produce an output file for each week, based on which the simulator will generate a new history file

%- the one-stage problem describes the problem of assigning nurses to shift within one week
%- nurses have multiple skills (head nurse, trainee, ...), but only one can apply in a single shift

%Straightforward: Assignment of workers to shifts


%!! Problem description
%---
%The components of this problem are the nurses (Agent) and shifts (TimedTask). 
%Each assignment consists of one shift and one nurse.
%Groups are used for role assignment and to determine whether a day is part of a weekend.
%---------
%\subsection{Example}
%Example taken from test data from INRC2 website:

This is an example for a scenario (global part of the problem description) from a test instance:
\begin{verbatim}
SCENARIO = n005w4

WEEKS = 4

SKILLS = 2
HeadNurse
Nurse

SHIFT_TYPES = 3
Early (2,5)
Late (2,3)
Night (4,5)

FORBIDDEN_SHIFT_TYPES_SUCCESSIONS
Early 0
Late 1 Early
Night 2 Early Late

CONTRACTS = 2
FullTime (15,22) (3,5) (2,3) 2 1
PartTime (7,11) (3,5) (3,5) 2 1

NURSES = 5
Patrick FullTime 2 HeadNurse Nurse 
Andrea FullTime 2 HeadNurse Nurse
Stefaan PartTime 2 HeadNurse Nurse
Sara PartTime 1 Nurse
Nguyen FullTime 1 Nurse
\end{verbatim}

%------------------------

Only one time unit is necessary: a counter of days since the start of the planning horizon.
Create groups: HeadNurse, Nurse, Early, Late, Night, Saturday1, Sunday1, Saturday2, Sunday2, Saturday3, Sunday3, Saturday4, Sunday4
Nurse can be used as an alias for Agent, Shift for TimedTask.

\begin{minted}{cpp}
using Nurse = Agent<std::string, std::string, std::string>;
using Shift = TimedTask<std::string, std::string, std::string, int>;
\end{minted}

Create each nurse, add the appropriate groups for roles. When reading from a file, this can be done in a loop; here only examplary for a single nurse:
\begin{minted}{cpp}
Nurse patrick("Patrick");
patrick.addGroup("HeadNurse");
patrick.addGroup("Nurse");
\end{minted}

There is only a single type of assignment for this problem. Since the problem is to assign nurses \textit{to} shifts, each shift will be fixed in one assignment.
\begin{minted}{cpp}
auto &asgn = inrc2.newAssignment(name);
auto &time = inrc2.newComponent(name, timeType);
time.addGroup(skill);
asgn.setFixed(timeSlot, time);
\end{minted}



Create a TimedTask for every role required for each shift. Add the appropriate groups for role requirements, shift type and day of the week.
Add these requirements to all Shifts $s$: 
\begin{minted}{cpp}
s.Require(shift.inGroup(HeadNurse) IMPLIES nurse.inGroup(HeadNurse));
s.Require(shift.inGroup(Nurse) IMPLIES nurse.inGroup(Nurse));
\end{minted}


Ideas to model the constraints as described in \cite{ceschia2019second} as rules:
\begin{description}
\item[H1. Single assignment per day]

Implies(SameGroup(timeslot, day), not(SameComponent(nurse)))
Implies(SameComponent(dayslot), not(SameComponent(nurse)))

internally:
timeslots are fixed, so just add illegal combinations directly

%A1.Simultaneous(A2) IMPLIES A1.Nurse != A2.Nurse, or
%A1.Shift.StartTime == A2.Shift.StartTime IMPLIES A1.Nurse != A2.Nurse

\item[H2. Under-staffing]
This is already enforced by making required tasks non-optional.

\item[H3. Illegal shift type successions]
For every illegal combination (prec, succ):

InGroup(A1.Shift, prec) AND InGroup(A2.Shift, succ) AND ImmediateSuccessor(A1.Shift, A2.Shift) IMPLIES A1.Nurse != A2.Nurse

Could list out not(SameComponent())

SameComponent()

\item[H4. Missing required skill]
Enforced by making required role tasks non-optional.

\item[S1. Insufficient staffing for optimal coverage]
Enforced by adding the appropriate weight (30) as penalty to optional shifts.

\item[S2. Consecutive assignments]

Define 

SameType(A1, A2) := (A1.shift.inGroup(Early) AND A2.shift.inGroup(Early)) OR (A1.shift.inGroup(Late) AND A2.shift.inGroup(Late)) OR (A1.shift.inGroup(Night) AND A2.shift.inGroup(Night))

MinConsecutive(A1.nurse == A2.nurse AND SameType(A1, A2)) >= MIN

\item[S3. Consecutive days off]

%MaxConsecutive(A1.nurse == A2.nurse AND SameType(A1, A2)) <= MAX

\item[S4. Preferences]

\item[S5. Complete week-end]

\item[S6. Total assignments (only evaluated at the end of the planning period)]

\item[S7. Total working week-ends (only evaluated at the end of the planning period)]


\end{description}

 

%\subsubsection{Translation}

%start with Scenario file: 
%\begin{enumerate}
%	\item Get scenario id for problem id
	
%	\item number of weeks and shifts: create for every task and every shift type a timedtask. Set shift type as group on the tasks.
	
%	\item Skills: determines the roles
	
%	\item For each shift type, a minimum and maximum number of consecutive assignments is given. These needs to be expressed as a constraint on every assignment 
	
%	\item For forbidden shift type successions: for every assignment and every nurse?? 
	
%	\item Contracts: define minimum and maximum number of assignments, consecutive work days, consecutive days off, maximum number of working weekends and complete weekends
	
%	\item Nurses: set contracts and skills. For each nurse, add an agent with the skills as roles. Contracts define additional constraints per nurse: 
%	\begin{enumerate}
%		\item Minimum number of assignments in planning horizon
%		\item Maximum number of assignments in planning horizon
%		\item Minimum number of consecutive working days
%		\item Maximum number of consecutive working days
%		\item Minimum number of consecutive days off
%		\item Maximum number of consecutive days off
%		\item Maximum number of working week-ends in the planning horizon
%		\item Whether weekends need to be complete (either fully free or fully worked)
%	\end{enumerate}
	
	
%	- "min/max in time" can be count bits in cropped bitvector!
	
%\end{enumerate}

%From Week Data: 


