\newcommand{\s}{\hspace{0.1cm}}

\subsection{Conditions}

\def\arraystretch{1.5}
\begin{tabular}{|p{0.5\textwidth} p{0.5\textwidth}|}
	
	\hline
	\multicolumn{2}{|c|}{Basic} \\
	\hline
	ComponentIs(AssignmentID, SlotID, Component) &	ComponentIn(AssignmentID, SlotID, Component...) \\	
	SameComponent(Assignments..., SlotID) & InGroup(assignment, slotID, groupID) \\
	TagEqual(Assignment, SlotID, TagID, Value) & \\
	TagEqualAbove(assignment, slotID, tagID, value) & TagEqualBelow(assignment, slotID, tagID, value) \\
	
	\hline
	\multicolumn{2}{|c|}{Boolean} \\
	\hline
	Not(Condition)	& \\
	And(Condition...)	&	Or(Condition...) \\
	Implies(Condition, Condition)	&	Iff(Condition, Condition) \\
	Xor(Condition, Condition)	& \\
	
	\hline
	\multicolumn{2}{|c|}{Ordered} \\
	\hline
	Simultaneous(TimedComponent...)	& \\
	Consecutive(TimedComponent1, TimeComponent2)	& \\
	
	\hline
	\multicolumn{2}{|c|}{Counting Unordered} \\
	\hline
	MaxAssignment(Assignment..., Condition...)	& 	MinAssignment(Assignment..., Condition...) \\
	
	\hline
	\multicolumn{2}{|c|}{Counting Ordered} \\
	\hline
	\multicolumn{2}{|l|}{MaxConsecutive(Assignment..., OrderedComponentSlot, Condition...)} \\
	\multicolumn{2}{|l|}{MinConsecutive(Assignment..., OrderedComponentSlot, Condition...)}	\\
	\multicolumn{2}{|l|}{MaxInSequence(Assignment..., OrderedComponentSlot, Condition...)}	\\
	\multicolumn{2}{|l|}{MinInSequence(Assignment..., OrderedComponentSlot, Condition...)}	\\
	
	\hline
	\multicolumn{2}{|c|}{???} \\
	\hline
	Only(Condition...) & \\
	Exclude(Condition...)	&	Include(Condition...) \\
	
	\hline

\end{tabular}
\def\arraystretch{1}

\subsubsection{Basic Conditions}

\begin{description}
\item[ComponentIs(Assignment, Slot, Value)]

\begin{tabular}{l l}
	\textbf{formal} & $ x_{a\text{,} s} = c_j $ \\
	\textbf{smt2} & $ (= \s x_{a\text{,}s} \s c_j) $
\end{tabular}


\item[ComponentIn(Assignment, Slot, Value...)]	

\begin{tabular}{l l}
	\textbf{formal} & \\
	\textbf{smt2} & 
\end{tabular}


$$ \bigvee_{c_j\in values} s_{a_i} = c_j $$



\item[SameComponent(Assignment..., Slot)]
For each combination of two assignments $(a_i, a_j), i\neq j$, add the constraint
$$ x_{a_i, slot} = component1 \wedge x_{a_j, slot} = component2 \implies component1 \neq component2 $$

\begin{tabular}{l l}
	\textbf{formal} & \\
	\textbf{smt2} & 
\end{tabular}

		
\item[InGroup(Assignment, Slot, Group)]
This condition can be encoded by a constraint that limits the domain of the slot variable to the component variables that fulfil the group condition.
Let $comps$ be the component variables corresponding to exactly those components   
$$ \bigvee_{c\in allComponents} $$



\item[TagEqual(assignment $a$, slot $s$, tag, value)]
Like InGroup, this condition can be handled by passing a constraint that limits the domain of assignment variable.
Let $comps$ be the component variables corresponding to exactly those components $c_i$ where $tag_{c_i} = value$.
$$ \bigvee_{c\in comps} x_{a,s} = c$$

\item[TagEqualAbove(assignment, slotID, tagID, value)]	
Let $comps$ be the component variables corresponding to exactly those components $c_i$ where $tag_{c_i} \geq value$.
$$ \bigvee_{c\in comps} x_{a,s} = c$$


\item[TagEqualBelow(assignment, slotID, tagID, value)]	
Let $comps$ be the component variables corresponding to exactly those components $c_i$ where $tag_{c_i} \leq value$.
$$ \bigvee_{c\in comps} x_{a,s} = c$$



\end{description}


\subsubsection{Composite Conditions}

\textit{Condition} used inside .. signifies the evaluation of the .

\begin{description}
\item[Not(Condition)]
$$ \neg Condition $$



\item[And(Condition...)]
$$ \bigwedge_{c\in Conditions} c$$

\begin{tabular}{l l}
	\textbf{formal} & \\
	\textbf{smt2} & 
\end{tabular}


\item[Or(Condition...)]
$$ \bigvee_{c\in Conditions} c$$


\item[Implies(Condition1, Condition2)]
$$ Condition1 \implies Condition2 $$

\begin{tabular}{l l}
	\textbf{formal} & \\
	\textbf{smt2} & 
\end{tabular}


\item[Xor(Condition1, Condition2)]
$$ (\neg Condition1 \wedge Condition2) \vee (Condition1 \wedge \neg Condition2) $$

\begin{tabular}{l l}
	\textbf{formal} & \\
	\textbf{smt2} & 
\end{tabular}


\item[Iff(Condition1, Condition2)]
$$ (Condition1 \implies Condition2) \wedge (Condition2 \implies Condition1) $$

\begin{tabular}{l l}
	\textbf{formal} & \\
	\textbf{smt2} & 
\end{tabular}


\item[MaxAssignment(Assignment..., Condition...)]
\item[MinAssignment(Assignment..., Condition...)]

Add a bitvector $bv_i$ of the length $\#Assignments$
For each assignment $a_k$, add a rule of the form
$$ \bigwedge_{c\in\text{Condition}} c(a_i) \implies extract(a_k) $$


%Then add the 
\cite{erkinger2017personnel} describe a declarative  , the number of set bits can be checked by 
$$ bv = bv \& (bv - 1) $$

If this constraint is violated, the penalty should be multiplied by the Hamming-Weight of the resulting vector.


\item[MaxConsecutive(Assignment..., OrderedComponentSlot, Condition...)]
\item[MinConsecutive(Assignment..., OrderedComponentSlot, Condition...)]

All given assignments must have a slot of the \textit{OrderedComponentSlot} type. This type needs to be an ordered type.
The $i$th bit of the generated bitvector corresponds to the $i$th assignment with respect to the given order. 
Assignments that are equivalent with respect to the given order are \textit{collapsed} into a bit that is set if the conditions hold for at least one of the equivalent assignments.
Let $Conditions(A)$ denote the conjunction of the given conditions resolved for the assignment $A$. For every given assignment, add:
$$ Conditions(A) \implies bv.extract(i) $$
As described in \cite{erkinger2017personnel}, the following equation will hold iff there exists no block of a length more than $max$:
$$0 = bv \& (bv >> 1) \& (bv >> 2) \& . . . \& (bv >> max)$$

The same paper describes a three-step algorithm to check minimal block length by first 

\item[Simultaneous(Assignment..., OrderedComponentSlot)]

Each of the given assignments needs to have a slot of the \textit{OrderedComponentSlot} type and that type needs to be ordered.



\item[Consecutive(Assignment..., OrderedComponentSlot)]

\item[Only(Assignment, Condition...)]



\end{description}


